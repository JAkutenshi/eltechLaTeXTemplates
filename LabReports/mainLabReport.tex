% Шаблон для отчетов пол лабораторным работам в СПбГЭТУ "ЛЭТИ"
% Ефремов М.А. 2017г
%=== Тип документа - статья, кегль 14пт.
\documentclass[14pt]{article}
%=== Настройка кодировок, шрифта и языка
\usepackage[utf8]{inputenc}
\usepackage{extsizes}
\usepackage[main=russian, english]{babel}
\usepackage[T2A, T1]{fontenc}
%=== Разметка документа
\usepackage{geometry} 
\geometry{
	a4paper, 
	top = 2cm,
	bottom = 2cm,
	left = 3cm,
	right = 1cm
}
%=== Форматирование текста
\usepackage {setspace}			% Интерлиньяж
\onehalfspacing					% 1.5 строки
\usepackage {indentfirst} 		% Красная строка с первого предложения
\setlength						% Отступ красной строки - 1.25см
	{\parindent}
	{1.25cm}	
\usepackage {titlesec}			% Форматирование заголовков
\titleformat					% Разделы
	{\section}
	[hang]
	{\normalfont\bfseries}
	{}
	{0pt}
	{}
\titlespacing
	{\section}
	{\parindent}
	{4ex}
	{0pt}
\titleformat					% Подразделы
	{\subsection}
	[hang]
	{\normalfont\bfseries\itshape}
	{\arabic{subsection}. }
	{0pt}
	{}
\titlespacing
	{\subsection}
	{\parindent}
	{4ex}
	{0pt}
%=== Минимизируем количество переносов
\usepackage {ragged2e}
\usepackage {microtype}
\tolerance = 500
\hyphenpenalty = 20000
\emergencystretch = 1cm
%=== Таблицы
\usepackage {tabularx}	% основной тип таблиц, выравнивание по ширине
\usepackage {longtable}	% для таблиц, не вмещающихся на одну страницу
\usepackage {multirow}	% для разбиения ячеек на несколько строк
\usepackage {multicol}	% на несколько колонок
%=== ^ до этого места - минимальная преамбула документа.
%=== Далее идут опциональные, но часто использущиеся пакеты,
%=== а так же написанные мной команды, чем-то упрощающие написание отчетов

%=== Работа с формулами
% Набор пакетов, сильно расширяющих возможности по набору формул
\usepackage{amsmath}
% добавляет специфические для русских  статей мат. символы вроде \leqslant
\usepackage{amssymb}
% добавляет окружения для теорем и лемм	
\usepackage{amsthm}				
\usepackage{mathtools}
% номера только для тех формул, на которые есть ссылки в тексте
\mathtoolsset{showonlyrefs=true}
%=== Работа с изображениями
\usepackage{graphicx}
%=== Работа с гиперссылками
\usepackage[unicode]{hyperref}
\hypersetup{
	colorlinks=true,
	urlcolor=blue,
	filecolor=green,
	linkcolor=red
}

\begin{document}
	\pagenumbering{gobble}
\clearpage
\begin{center}	
	МИНОБРНАУКИ РОССИИ\\
	САНКТ-ПЕТЕРБУРГСКИЙ ГОСУДАРСТВЕННЫЙ\\
	ЭЛЕКТРОТЕХНИЧЕСКИЙ УНИВЕРСИТЕТ\\
	«ЛЭТИ» ИМ. В.И. УЛЬЯНОВА (ЛЕНИНА)\\
	Кафедра МО ЭВМ

	\vspace{54mm}

	ОТЧЕТ\\
	по лабораторной работе №0 \\
	по дисциплине «Наименование дисциплины» \\
	Тема: Наименование темы \\

	\vspace{65mm}

	\def\arraystretch{1.5}
	\begin{tabularx}{\textwidth}{ >{\hsize=7cm}X >{\hsize=4cm}X  >{\centering\arraybackslash}X }
		Студент гр. 2304 & & Ефремов М.А. \\ \cline{2-2}
		Преподаватель & & Фамилия И.О. \\ \cline{2-2}
	\end{tabularx}
	\def\arraystretch{1}

	\vfill
	Санкт-Петербург\\
	2017
\end{center}
\newpage
\pagenumbering{arabic}
\setcounter{page}{2}
	\section{Цель работы.}
	Приводится цель работы в соответствии с методическими указаниями.

	\section{Основные теоретические положения.}
	Раздел выполняется в соответствии с указаниями преподавателя.
	
	В разделе может быть приведено описание исследуемых физических явлений (с иллюстрациями), основные теоретические положения (в том числе – математический аппарат, описывающий исследуемые явления), схемы измерений, сведения об используемом при проведении работы лабораторном оборудовании.
	
	\section{Экспериментальные результаты.}
	Приводятся экспериментальные данные, в том числе результаты моделирования (обычно в виде таблиц).
	
	\section{Обработка результатов эксперимента.}
	Приводятся результаты обработки экспериментальных данных, результаты расчетов, графики полученных зависимостей, иные требуемые методическими указаниями данные.
	
	\section{Выводы.}
	Оценивается степень соответствия полученных результатов расчетов и экспериментов с теоретическими данными. 
	
	Дается объяснение полученных в ходе работы зависимостей и результатов. \newline

	\textit{Студенты имеют право оформлять отчет как в рукописном варианте, так и использовать для оформления и печати ЭВМ и МФУ.	}
	
\end{document}